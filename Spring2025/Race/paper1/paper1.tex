% Options for packages loaded elsewhere
\PassOptionsToPackage{unicode}{hyperref}
\PassOptionsToPackage{hyphens}{url}
%

\documentclass[
]{article}
\usepackage{amsmath,amssymb}
\usepackage{iftex}
\ifPDFTeX
  \usepackage[T1]{fontenc}
  \usepackage[utf8]{inputenc}
  \usepackage{textcomp} % provide euro and other symbols
\else % if luatex or xetex
  \usepackage{unicode-math} % this also loads fontspec
  \defaultfontfeatures{Scale=MatchLowercase}
  \defaultfontfeatures[\sfdefault]{Ligatures=TeX, Scale=1}
\fi
\usepackage{lmodern}
\usepackage{times}
\ifPDFTeX\else
  % xetex/luatex font selection
\fi
% Use upquote if available, for straight quotes in verbatim environments
\IfFileExists{upquote.sty}{\usepackage{upquote}}{}
\IfFileExists{microtype.sty}{% use microtype if available
  \usepackage[]{microtype}
  \UseMicrotypeSet[protrusion]{basicmath} % disable protrusion for tt fonts
}{}
\makeatletter
\@ifundefined{KOMAClassName}{% if non-KOMA class
  \IfFileExists{parskip.sty}{%
    \usepackage{parskip}
  }{% else
    \setlength{\parindent}{0pt}
    \setlength{\parskip}{6pt plus 2pt minus 1pt}}
}{% if KOMA class
  \KOMAoptions{parskip=half}}
\makeatother
\usepackage{xcolor}
\setlength{\emergencystretch}{3em} % prevent overfull lines
\providecommand{\tightlist}{%
  \setlength{\itemsep}{0pt}\setlength{\parskip}{0pt}}
\setcounter{secnumdepth}{-\maxdimen} % remove section numbering
\ifLuaTeX
  \usepackage{selnolig}  % disable illegal ligatures
\fi
\usepackage{bookmark}
\IfFileExists{xurl.sty}{\usepackage{xurl}}{} % add URL line breaks if available
\urlstyle{same}
\hypersetup{
  hidelinks,
  pdfcreator={LaTeX via pandoc}}
\pagestyle{headings}

\author{Nate Harris}
\date{2/28/2025}


\usepackage{fancyhdr}
    \fancyfoot{}
    \rhead{Harris \thepage}
    \pagestyle{fancy}


\linespread{2}
\setlength{\parindent}{20pt}

\begin{document}



\noindent Nate Harris

\noindent Professor Richard Porten

\noindent English 0934

\noindent 18 Feb, 2025


\par
\emph{Minor Feelings} by Cathy Park Hong discusses what it means to be
Asian American. In the first chapter ``United'' Hong describes the Asian
American identity through her own lived experience. Through this Hong shows what it means
to be an ``Other'' in this country specifically, a Korean ``Other''.
There is an inherent wrongness to how Americans are socialized around
race. This wrongness manifests in roadblocks in Hong's life, such as in
her pursuit of a poetry Masters. In both direct and indirect ways Hong
paints the lasting anxiety related to not being white.
Hong describes that whiteness is an impossibility.
She describes how an Asian person is pushed to make themselves smaller.
She sees this in her father and in the man who was abused and beaten when being forced to leave his United flight, David Dao.
She sees that no matter the adaptation taken to the system, the Asian American is still taken advantage of by those in power.
\par
It is important to this writing to empathize with Hong as a person.
In ending her description of her anxious mindset in the first few pages Hong interrupts her pace of writing with these lines:
``I could not write nor could I socialize and carry on a conversation, I was the child again. The child who could not speak English'' (Hong, 5)
This line, especially after empathizing with Hong's struggles to find peace, hits hard.
This introduction gives Hong's reason for writing this, not for the sake of anthropology, but how her fears and anxieties, who she is as a person are all in some way directly tied to her identity as an Asian American.
This captures the otherness Hong experiences in just two sentences.
These insecurities and a feeling of being an outsider in this country are deeply intertwined. I think it also captures Hong's identity as a poet, trying to capture those emotions in prose.
\par
While I can't directly relate to why Hong feels this way I understand the feeling.
Nagging anxiety, knowing nothing gold can stay, a fear and inability to exist in a social space "properly". 
That kind of feeling stays with you, a feeling of dis-belonging. 
It seems reasonable that being an outsider due to racial identity could lead to these feelings.
People desire social bonds.
Hong draws a salient connection between this depression and the difficulty our society has with promoting this bonding;
or really, the difficulty one has in creating these bonds in American society.
\par 
In response to this pain, Hong seeks out therapy.
To save time on explanation or, as stated later, maybe to find some \emph{jeong}-- "instantaneous deep connection"(Hong, 22) Hong seeks out a Korean therapist. 
In using the term, ``jeong'' Hong highlights a perceived ``Korean-ness'' to her depression, or maybe she feels her depression is one only Koreans could truly relate to.
Though this may only have been realized in post as the term ``jeong'' is only used after Hong is rejected by the therapist and is recounting it with maybe some regret.
This implies that Hong's reacting so strongly was not at the patient therapist bond but a perceived intimate and cultural bond, which was rejected.
In reflecting Hong proposes that maybe in searching for that jeong she was attempting to shortcut the long process that would be understanding herself.
Maybe she suggests here that not all problems she faced are inherently Korean and in assuming, she prolonged her depression.
Quoting her friend, ``... it's too easy to assume everything dysfunctional about your family is cultural. Sometimes you to explain your experiences in order to understand them yourself''
Hong may be suggesting that how one internalizes race could cloud their vision of themselves.
This reminds me of the complaints against Roth discussed in class that depict some Jewish characters as less than the moral standard.
\emph{Is one \_\_\_\_\_ because they are Jewish, or because they are human?}
In being a member of a racial group you may assume yourself to be incapable of or susceptible to certain behavior, whether moral or amoral, that anyone can experience or do.
Certain stereotypes and conceptions not only be used by the prejudiced and hateful to attack but in internalizing them they may be used against oneself to restrict.
In assuming that only a Korean therapist could understand, Hong found difficulty and anger when she couldn't.
This likely harmed both Hong and the therapist.
\par
Racial violence experienced by Hong is different from that of her father.
She feels it primarily through social means, through the words of classmate, conditional love growing up and feeling the society doesn't care about her.
``My confidence was impoverished from a lifelong diet of conditional love and society who thinks I am as interchangeable as lint.'' (Hong, 8)
Her father experienced it both physically and financially. 
``Ryder fired him instead of giving him workman's comp because they knew he couldn't do anything about it'' (Hong, 12)
This was after Hong's father's leg was broken at work.
Hong writes how her father carried this resentment onwards.  
Hong also writes how this is often not discussed when talking bout the model minority.
The immigrant parent is just expected to work without concept of race in order to provide, maybe to attain some idea of the ``American Dream''.
This does not apply for Hong's father, he holds onto how being Asian American-- not using Korean here as the generalization is what is weaponized-- exposes one to harm. 
When his daughter is heading to a largely white university he advises:
``Don't ever make an illegal U-turn here because they will see you are an Asian driving badly.''(Hong, 13).
Not to say one should drive badly but in warning his daughter the Asian driver stereotypes it shows how he feels he must constantly be vigilant of whether he is meeting the expectation of his race or not.
He understands that he or his daughter may be greater punished for this small moment of dis-legal driving. 
\par
I think Hong paints how racial trauma may be internalized and passed down for Asian Americans.
There is almost a banality to how her father speaks on the matter.
As if it were common sense that she must smile and be polite to her newfound white neighborhood or can never drive in a suboptimal fashion. 
Which while not bad advice for fitting in there is an undertone of expected violence if these rules break that is not expected of a white person.
You must do these things not to be polite, you must do them for your safety.
Which, given the earlier context of her roommate's father emphatically claiming he was holding a gun against her father's former countrymen, doesn't seem too ill-founded.
I think section demonstrates the pressure to perform as a model minority and how it forces an anxiety on to people.
\par
This leads Hong into the main topic of the writing, the forced removal of David Dao from that United Flight.
In expressing the thoughts of her father on Asian-ness in America Hong then goes to how Dao maybe had shared the same belief.
She expresses how she empathizes with this man and feels that her father may have been the body struck and bloody.
In describing Dao, Hong writes, ``It wasn't just that he was the same age as our fathers. It was also his trim and discreet appearance that made him familiar.'' (Hong, 26) 
Hong states not really appearance but how he carried himself is what made him seem like Her father-- a sentiment she shared with other Asian friends.
Here Dao is emblematic of the abuse people like her father would receive, similar to how her father was denied workman's comp, Dao was not given the respect one would expect as an American in their own country.
No matter how polite they still would be picked over whites for inconvenience, then violence.
\par
Hong captures here the escalation of abused ``Otherness'', how it is weaponized, how it turns to violence.
Going on to write about the falsehood of the model minority and to the absorbing monster of whiteness in America.
I agree with Hong on that Asian's are not up next to be white as being white even is somewhat of an impossibility.
There are no true hard boundaries on race or something scientific that classifies people as such.
Given that, being perceived as Asian in some way exposes Hong to some prejudice and danger.
Not to say that's all it does as obviously being a part of a cultural group helps you bond with people of similar experience-- Hong mentions Asian friends, and she grew up in Koreatown.
I think Hong dispels the myth that Asian Americans have it easy or have it ``white''. 
She gives a succinct and easy to understand both practically and emotionally view of her life and Asian American-hood.
I think in reading this one gains more empathy and appreciation for the Asian American experience.
Hong expresses that it is hard to empathize with those one doesn't know well, ``What did I know about being a Vietnamese teenage boy who spent all of his free hours working at a nail salon? I knew nothing.''
And I can't pretend I know what it means to truly be from Korea, or anywhere but my hometown. But I understand the fear Hong feels when she saw David Dao being assaulted on that plane better than before reading. Hong cultivates empathy with her writing which is in desperate need if one intends to create a better world. 

\center{Works Cited}


Hong, Cathy Park. “United.” Minor Feelings, One World, New York, New York, 2021, pp. 4–28. 
\end{document}
